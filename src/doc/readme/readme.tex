%
%  Copyright 2008-2014 Iowa State University
% 
%  This file is part of Mantis.
%  
%  Mantis is free software: you can redistribute it and/or modify
%  it under the terms of the GNU General Public License as published by
%  the Free Software Foundation, either version 3 of the License, or
%  (at your option) any later version.
%  
%  Mantis is distributed in the hope that it will be useful,
%  but WITHOUT ANY WARRANTY; without even the implied warranty of
%  MERCHANTABILITY or FITNESS FOR A PARTICULAR PURPOSE.  See the
%  GNU General Public License for more details.
%  
%  You should have received a copy of the GNU General Public License
%  along with Mantis.  If not, see <http://www.gnu.org/licenses/>.
% 
%  Author: Laura Ekstrand (ldmil@iastate.edu)
%

\documentclass[letterpaper,11pt]{article}
\usepackage{amsmath}
\usepackage[top=1.5in, bottom=1in, left=1in, right=1in]{geometry}
\usepackage{amsthm}
\usepackage{amssymb}
\usepackage{leftidx}
\usepackage{hyperref}

\title{ToolMarkSimulator Readme Document}
\author{Laura Ekstrand}
\date{\today}
\begin{document}
\maketitle

\section{Installing Qwt}
\begin{itemize}
\item Go to \url{http://sourceforge.net/projects/qwt/}. Click on the
Files tab (don't just click the download link; they're not good about updating
it  to the latest version). Click on the qwt folder and then the latest
release folder (as of now use version 6.0.1).  Download the zip file. (Linux
users might be able to bypass this step and the next two steps by downloading
a binary from the package manager -- but (very important!) check to make sure it is the latest
version of Qwt.)

\item Extract the files to a folder you have access to.  I recommend a folder
called build in your home directory.
\item Build qwt:  
If you are on Linux, open a terminal and go to the qwt folder
that you extracted. Type

{\tt qmake

make

make install.}

If you are on Windows, open the .pro file in the recently extracted folder with
QtCreator.  Choose a build target of Desktop with shadow building checked.
Build the project. When it successfully builds, close QtCreator.
Open the MSVC terminal. Type

{\tt nmake

nmake install.}

\item  Set the configuration features: (This is described here, for reference:
\url
{http://doc.qt.nokia.com/4.7-snapshot/qmake-advanced-usage.html#adding-new-configuration-features})



Go to the directory where Qwt is installed.  On Linux, I think this is under
usr/. On Windows, it will be on C:. Find the qwt.prf file (it should be in the
features folder; careful, Windows mistakenly calls it a PICS Resource File or something
like that). Copy or otherwise remember the path to this file.

On Linux, go to your home directory and edit your .bashrc file (or the config
file for whatever shell you are using).  On Windows, go to Start Menu
$\rightarrow$ Computer
$\rightarrow$ System Properties $\rightarrow$ Advanced? $\rightarrow$ Environment Variables.  In either case, add the environment variable
{\tt QMAKEFEATURES = [path to qwt.prf]}.

\item Now that the QMAKEFEATURES environment variable is set, the {\tt CONFIG
+= qwt}
line in your .pro file should work to add in the qwt library.  Of course, you
still need to include your qwt headers in your code.
\end{itemize}

\end{document}

